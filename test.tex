\documentclass[codepkg=listings,theme=fancy,twoside]{course-report}

\csrepset{
    style = {
        graphics-path = {{./figures/}, {./figure/}},
        punct = CCT,
        font = pagella,
        math-font = xcharter,
        cjk-font = sourcefz,                           % 设置中文字体为思源和方正字体,若未安装请设置为ctex
        today = big,                                   % 大写日期
        enname,                                        % 英文标题
        fullwidth-stop                                 % 英文全角句点
    },
    info = {
        csnumber = 123,
        cstype = 专业必修,
        title = \LaTeX{} 课程报告模板,
        subtitle = 副标题,
        academy = 测绘学院,
        major = 导航工程,
        badge = logo.pdf,
        class = 导航3班,
        name = TAB,
        date = \today,
        addinfo = {
            { 添加项1, 测试1 },
            { 添加项2, 测试2 }
        }
    },
    code/language = C++,
    % bib = {
    %     bibliography = {ref/refs.bib , ref/thu.bib},
    %     backend = bibtex,
    % }
}

\usepackage{zhlipsum,lipsum}

\lstset{
    keywordstyle = \color{red},
}

\begin{document}

\chapter{测试章节}
\section{字体测试}

英文罗马字族:\\
\lipsum[1]

英文意大利字形:\\
{\itshape \lipsum[1] }

英文倾斜字形(取决于该英文字体是否有倾斜字形,没有则与意大利字形相同):\\
{\slshape \lipsum[1]}

英文粗体:\\
{\bfseries \lipsum[1]}

中文罗马字族:\\
\zhlipsum[1]

中文意大利字形:\\
{\itshape \zhlipsum[1] }

中文粗体:\\
{\bfseries \zhlipsum[1] }

{\songti 宋体} {\kaishu 楷书} {\fangsong 仿宋} {\heiti 黑体}

数学字体:
\[
    \int_{f^{ - 1}([a,b])} g(x)\,\mathrm{d}x_1\mathrm{d}x_2\cdots\mathrm{d}x_n =\int_a^b\,\mathrm{d}t\int_{f^{ - 1}(t)}\frac{g}{\left\|\nabla f\right\|}\,\mathrm{d}\sigma.
\]

\subsection{测试小节}

\zhlipsum[1-2]\zhlipsum[1]
\begin{codebox}{注释 test}\label{box:1.1}
int i = 1 
int i = 1    
int i = 1    
int i = 1    
int i = 1    
\end{codebox}
参考文献\codeinline{int i = 1}.


\begin{codebox}{注释 test}\label{box:1.2}
    int i = 1    
\end{codebox}


\chapter{测试章节二}


\appendix
\chapter{附录}

\nocite{*}


\end{document}