\documentclass[codepkg = minted, titlestyle = lralign]{course-report}

\csrepset{
    style = {
        graphics-path = {{figures/}},
        punct         = CCT,
        font          = pagella,
        math-font     = pagella,
        today         = big,
        enname
    },
    info = {
        csnumber   = 123456789,
        cstype     = 专业必修,
        title      = \LaTeX 课程报告模板,
        subtitle   = 副标题,
        department = 测绘学院,
        major      = 导航工程,
        badge      = logo.pdf,
        author     = Eliauk,
        id         = 202020212022,
        date       = \today,
        addinfo    = {
          {组织, WHUTUG}
        }
    },
    code = {
      language = cpp,
      boxstyle = simple
    },
    bib = {
      bibliography = {ref/refs.bib, ref/thu.bib}
    }
}

\usepackage{zhlipsum,lipsum}



\begin{document}

\maketitle

\frontmatter

\tableofcontents

\mainmatter

\chapter{中文测试}

\section{测试节}

\zhlipsum[1]

\subsection{测试小节}

{
  \bfseries
  \zhlipsum[1-4][name = nanshanjing]
}


\subsubsection{测试小小节}

{
  \itshape  
  \zhlipsum[1-3][name = xiangyu]
}



\chapter{英文测试}

\section{测试节}

\zhlipsum[1]
\begin{english}
  \lipsum[1]
\end{english}
\zhlipsum[1]


\subsection{测试小节}

\zhlipsum[1]

\begin{english}
  \itshape
  \lipsum[2-5]
\end{english}

\zhlipsum[1]


\subsubsection{测试小小节}

\zhlipsum[1]

\begin{english}
  \bfseries
  \lipsum[6-7]
\end{english}

\zhlipsum[1]


\chapter{代码框测试}

\section{测试节}

\begin{codebox}[
  minted language = csharp
]{跳秒表。}
public static class LeapSeconds
{
    public readonly static Dictionary<DateTime, int> GpsTimeTable = new() 
    {
        {new(2016, 12, 31, 23, 59, 59), 18},
        {new(2015, 6, 30, 23, 59, 59), 17},   
        {new(2012, 6, 30, 23, 59, 59), 16},    
        {new(2008, 12, 31, 23, 59, 59), 15},    
        {new(2005, 12, 31, 23, 59, 59), 14},    
        {new(1998, 12, 31, 23, 59, 59), 13},    
        {new(1997, 6, 30, 23, 59, 59), 12},    
        {new(1995, 12, 31, 23, 59, 59), 11},    
        {new(1994, 6, 30, 23, 59, 59), 10},    
        {new(1993, 6, 30, 23, 59, 59), 9},    
        {new(1992, 6, 30, 23, 59, 59), 8},    
        {new(1990, 12, 31, 23, 59, 59), 7},    
        {new(1989, 12, 31, 23, 59, 59), 6},    
        {new(1987, 12, 31, 23, 59, 59), 5},    
        {new(1985, 6, 30, 23, 59, 59), 4},    
        {new(1983, 6, 30, 23, 59, 59), 3},    
        {new(1982, 6, 30, 23, 59, 59), 2},    
        {new(1981, 6, 30, 23, 59, 59), 1},   
    };
}
\end{codebox}

\begin{codebox}
public static double Solve(double initialValue, double tolerance, int maxIterationNum, Func<double, double> funtion, Func<double, double> derivativeFunction)
{
    double xPre = initialValue;
    double xCur = initialValue - funtion(initialValue) / derivativeFunction(initialValue);
    int count = 0;
    while (System.Math.Abs(xCur - xPre) > tolerance)
    {
        xPre = xCur;
        count++;
        xCur -= funtion(xCur) / derivativeFunction(xCur);
        if (count > maxIterationNum)
            break;
    }
    return xCur;
}
\end{codebox}

\begin{codebox*}{
  大地椭球。
}
template<CoordinateSystem T>
struct Elliposid;

template<>
struct Elliposid<CoordinateSystem::CGCS2000>
{
    constexpr static double semi_minor     = 6356752.3141403558;
    constexpr static double semi_major     = 6378137;
    constexpr static double oblateness     = 1.0 / 298.2572221010042;
    constexpr static double eccentricity_1 = 0.081819191042811;
    constexpr static double eccentricity_2 = 0.0820944381519236;
};
\end{codebox*}

\begin{codebox*}
template<class T>
class ErrorDetector;

template<>
class ErrorDetector<Range>
{
public:
    ErrorDetector(): preRange() { }
    void ErrorDetect(Range& curRange);
private:
    Range preRange;
};
\end{codebox*}

\backmatter

\nocite{*}

\makebibliography

\end{document}